
Large pre-trained language models have been shown to store factual knowledge in their parameters, and achieve state-of-the-art results when fine-tuned on downstream NLP tasks.
However, their ability to access and precisely manipulate knowledge is still limited, and hence on knowledge-intensive tasks, their performance lags behind task-specific architectures.
%
Additionally, providing provenance for their decisions and updating their world knowledge remain open research problems.
Pre-trained models with a differentiable access mechanism to explicit non-parametric memory
% 
have so far been only investigated for extractive downstream tasks.
%  
We explore a general-purpose fine-tuning recipe for retrieval-augmented generation (RAG)  --- models which combine pre-trained parametric and non-parametric memory for language generation. 
We introduce RAG models where the parametric memory is a pre-trained seq2seq model and the non-parametric memory is a dense vector index of Wikipedia, accessed with a pre-trained neural retriever. 
% 
We compare two RAG formulations, one which conditions on the same retrieved passages across the whole generated sequence, and another which can use different passages per token.
We fine-tune and evaluate our models on a wide range of knowledge-intensive NLP tasks and set the state of the art on three open domain QA tasks, outperforming parametric seq2seq models and task-specific retrieve-and-extract architectures.  
For language generation tasks, we find that RAG models generate more specific, diverse and factual language than a state-of-the-art parametric-only seq2seq baseline. 
% 