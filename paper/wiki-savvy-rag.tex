\pdfoutput=1

\documentclass[11pt]{article}

% Remove the "review" option to generate the final version.
\usepackage{WIKI-SAVVY-RAG}
\usepackage{times}
\usepackage{latexsym}
\usepackage[T1]{fontenc}
\usepackage[utf8]{inputenc}
\usepackage{microtype}
\usepackage{inconsolata}


% If the title and author information does not fit in the area allocated, uncomment the following
%
%\setlength\titlebox{<dim>}
%
% and set <dim> to something 5cm or larger.

\title{Wikipedia-Savvy-RAG: Wikipedia-backed RAG system expert about STEM subjects}  

\author{Dario Filatrella \and Michele Palma \and Mattia Scardecchia \and Federico Zarantonello}

\begin{document}
\maketitle
\begin{abstract}
We present Wikipedia-Savvy-RAG, a Retrieval-Augmented Generation (RAG) system 
fine-tuned for STEM domain discussions. Our system features a chat interface that 
enhances responses using retrieved Wikipedia passages.  
We assessed the system using STEM question-answering benchmarks from the MMLU dataset, 
fine-tuning the LLM for improved performance. 

\end{abstract}

\section{Introduction}

\textit{Lorem ipsum dolor sit amet, consectetur adipiscing elit. Sed nec purus eget
nunc. Donec tincidunt, nunc in volutpat tempus, nulla justo varius lacus, vel
suscipit orci mi eget justo. Nulla sit amet magna in odio aliquam suscipit.
Vestibulum ante ipsum primis in faucibus orci luctus et ultrices posuere cubilia
Curae; Pellentesque suscipit tincidunt magna non mollis. Fusce tempus, nulla non
molestie luctus, metus libero euismod augue, a mollis lacus purus ac ex.}

\section{Related Works}

\textit{Lorem ipsum dolor sit amet, consectetur adipiscing elit. Sed nec purus eget
nunc. Donec tincidunt, nunc in volutpat tempus, nulla justo varius lacus, vel
suscipit orci mi eget justo. Nulla sit amet magna in odio aliquam suscipit.
Vestibulum ante ipsum primis in faucibus orci luctus et ultrices posuere cubilia
Curae; Pellentesque suscipit tincidunt magna non mollis. Fusce tempus, nulla non
molestie luctus, metus libero euismod augue, a mollis lacus purus ac ex.}

\section{Vector Database}

One of the main components of most of retrieval-based systems is the vector database.
In our case, we built a vector database of Wikipedia passages to be used by the RAG model.

\subsection{Cleaning Wikipedia}

\textit{Remember to talk about SQLite database}

\subsection{Embeddings}

\textit{Lorem ipsum dolor sit amet, consectetur adipiscing elit. Sed nec purus eget
nunc. Donec tincidunt, nunc in volutpat tempus, nulla justo varius lacus, vel
suscipit orci mi eget justo. Nulla sit amet magna in odio aliquam suscipit.}

\subsection{Faiss}

To handle efficiently the retrieval of the 13 million embedded chunks, we used the Faiss library. 
Faiss is a library for efficient similarity search and clustering of dense vectors. 
It offers different indexing methods. Flat uses simple brute-force search, 
IVF uses clustering to reduce the number of comparisons,
and HNSW creates a multi-layered graph structure where each layer is a simplified, 
navigable small world network. 

Faiss also offers different quantization techniques to reduce the memory usage of the embeddings, 
the two we considered are the product quantization and the scalar quantization. 
The product quantization splits the vectors into sub-vectors and quantizes each sub-vector separately,
while the scalar quantization quantizes each dimension of the vectors separately.

\section{RAG}

\textit{Lorem ipsum dolor sit amet, consectetur adipiscing elit. Sed nec purus eget
nunc. Donec tincidunt, nunc in volutpat tempus, nulla justo varius lacus, vel
suscipit orci mi eget justo. Nulla sit amet magna in odio aliquam suscipit.}

\subsection{LLM}

\textit{Lorem ipsum dolor sit amet, consectetur adipiscing elit. Sed nec purus eget
nunc. Donec tincidunt, nunc in volutpat tempus, nulla justo varius lacus, vel
suscipit orci mi eget justo. Nulla sit amet magna in odio aliquam suscipit.}

\subsection{Generation}

\textit{Talk about the two ways to generate the answer: naive and REPLUG}

\section{Finetuning}

\textit{Lorem ipsum dolor sit amet, consectetur adipiscing elit. Sed nec purus eget
nunc. Donec tincidunt, nunc in volutpat tempus, nulla justo varius lacus, vel
suscipit orci mi eget justo. Nulla sit amet magna in odio aliquam suscipit.}

\subsection{Data}

\textit{Lorem ipsum dolor sit amet, consectetur adipiscing elit. Sed nec purus eget
nunc. Donec tincidunt, nunc in volutpat tempus, nulla justo varius lacus, vel
suscipit orci mi eget justo. Nulla sit amet magna in odio aliquam suscipit.}

\subsection{Training}

\textit{Lorem ipsum dolor sit amet, consectetur adipiscing elit. Sed nec purus eget
nunc. Donec tincidunt, nunc in volutpat tempus, nulla justo varius lacus, vel
suscipit orci mi eget justo. Nulla sit amet magna in odio aliquam suscipit.}

\section{Demo}

To demonstrate the capabilities of our system, we have developed a Streamlit web application
that allows users to interact with the model in real-time with a chat-like interface. 
The application lets users ask questions about STEM subjects, and the model will provide answers 
based on the Wikipedia passages it retrieves from its knowledge base. 
The user can decide the number of passages to retrieve, the decoding strategy (greedy, top-k, top-p), 
the inference type (naive or REPLUG) and other configuration parameters.
Refer to our GitHub repository for more information on the application.

\subsection{Arxiv}

\textit{???????????????????????????}

\section{Benchmark}

\textit{Lorem ipsum dolor sit amet, consectetur adipiscing elit. Sed nec purus eget
nunc. Donec tincidunt, nunc in volutpat tempus, nulla justo varius lacus, vel
suscipit orci mi eget justo. Nulla sit amet magna in odio aliquam suscipit.}

\subsection{Faiss index}

To choose the best Faiss index for our system, we compared the performance of 
different indexes. The task was to retrieve the K most similar passages to a
given query coming from the MMLU dataset.
The parameters we considered were the recall, the query time, 
and the memory usage. 
The recall was computed as the intersection measure between the retrieved 
passages and the ground truth. 
The variations we considered were the index type (Flat, IVF, HNSW), the number 
of clusters for the IVF index, the quantization technique (product, scalar), and
the amount of quantization bits.

\subsection{MMLU}

\section{Results}

\textit{Lorem ipsum dolor sit amet, consectetur adipiscing elit. Sed nec purus eget
nunc. Donec tincidunt, nunc in volutpat tempus, nulla justo varius lacus, vel
suscipit orci mi eget justo. Nulla sit amet magna in odio aliquam suscipit.}

\subsection{Training}

\textit{Lorem ipsum dolor sit amet, consectetur adipiscing elit. Sed nec purus eget
nunc. Donec tincidunt, nunc in volutpat tempus, nulla justo varius lacus, vel
suscipit orci mi eget justo. Nulla sit amet magna in odio aliquam suscipit.}

\subsection{Faiss indexes}

We measured the recall of the indexes on k=1, 10, 50 and 100. The query time was 
the time it took to retrieve the 100 most similar passages for each one of the first 300
queries in the MMLU dataset with STEM subject. 
For the memory usage, we considered the size of the index in disk, which acts as a lower
bound for the memory usage.
The most interesting results are shown in Table \ref{tab:faiss-index}, where the indexes 
are the following:

% ordered list with numbers
\begin{enumerate}
    \item Flat index with 4 bits scalar quantization.
    \item Flat index with 8 bits scalar quantization.
    \item Flat index with 128 bits product quantization.
    \item Index with 256 bits product quantization, IVF with 1000 centroids
    and HNSW32 as coarse quantizer.
    \item HNSW index with 16 bits scalar quantization.
\end{enumerate}

\begin{table}[h]
\centering
\begin{tabular}{|c|c|c|c|}
\hline
Index & Recall & Query Time & Memory \\
\hline
1 & 0.832 & 03:53.59 & 2,609 \\
2 & 0.9873 & 03:29.50 & 5,219 \\
3 & 0.781 & 01:55.76 & 1,740 \\
4 & 0.4643 & 00:00.03 & 1,856 \\
5 & 0.661 & 00:00.11 & 4,570 \\
\hline
\end{tabular}
\caption{Faiss index comparison}
\label{tab:faiss-index}
\end{table}

\subsection{MMLU}
\textit{Lorem ipsum dolor sit amet, consectetur adipiscing elit. Sed nec purus eget
nunc. Donec tincidunt, nunc in volutpat tempus, nulla justo varius lacus, vel
suscipit orci mi eget justo. Nulla sit amet magna in odio aliquam suscipit.}

\section{Conclusion}
We developed, fine-tuned, and evaluated a Retrieval-Augmented Generation (RAG) 
system for STEM domain discussions. Our application features a chat interface 
that enhances responses using retrieved Wikipedia passages. We processed the 
English Wikipedia to build a vector database of STEM-focused semantic 
embeddings. 
We assessed the system using STEM question-answering benchmarks from the MMLU 
dataset, fine-tuning the LLM for improved performance. Evaluations compared 
performance with and without RAG, before and after fine-tuning.
Results show significant accuracy improvements, highlighting RAG's potential 
in specialized domains like STEM.

\bibliography{wiki-savvy-rag}
\bibliographystyle{acl_natbib}

\appendix

\section{Appendix}
\label{sec:appendix}

This is a section in the appendix.

\end{document}
